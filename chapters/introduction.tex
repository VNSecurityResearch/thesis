\chapter[Introduction]{Introduction}

In this chapter, we explain the definitions along with the current state of digital forensics and some contexts of today computer system and the current state we are in.

Before we get into the proposal, we shall understand the following terminologies as written below:

\begin{itemize}
  \item \textbf{Hacker}.
    Person who commits illegal activities to computer systems by exploiting bugs, or by writing illegal software that aims to damage or corrupt system, or by collecting sensitive, classified information.
  \item \textbf{Malware}.
    A type of computer software that does illegal activities such as stealing information; damaging or corrupting system; ransoming system owner; etc..
  \item \textbf{Trojans}.
    A type of computer software that collects computer information and activties without user's permission and/or acknowledgement.
    A digital asset used in place of cash or banking money, hardened by cryptographic security.
  \item \textbf{File signature}.
    Data used to identify or verify the content of a file. Commonly bytes in file and file checksum.
\end{itemize}

\section[Motivation]{Motivation}

Throughout the years of computer development, computers have become a standard method for humans around the world to study, work and entertain. Most activities of our daily lives involve computers. Individuals across the globe have created systems running on computers to assist them in doing common and complex tasks. However, this somehow influenced other people to commit harmful activities. Hackers have been creating malware damaging and stealing information. Furthermore, lately along with the trend of cryptocurrency, while ordinary people commit their investment by using crypto mining machines, hackers, on the other hand, create sophisticated malware (cryptojacking malware) that stealthily installed on a victim machine and mine cryptocurrency without the victim's acknowledgement.

Security researchers have been struggling to find ways to mitigate the gaining rate of attacks. However, it was never close to perfection. We rely on file signature database for filtering and often miss out new one, thus, it is highly vulnerable to the newer class of malware. To counterattack these new malware, we perform digital forensics when an attack happens. Digital forensics which as described by The Forensics Research Workshop I \cite{roadmap}:

\say{The use of scientifically derived and proven methods toward the preservation, collection, validation, identification, analysis, interpretation, documentation and presentation of digital evidence derived from digital sources for the purpose of facilitating or furthering the reconstruction of events found to be criminal, or helping to anticipate unauthorized actions shown to be disruptive to planned operations.}

% https://www.mcafee.com/enterprise/en-us/assets/infographics/infographic-threats-report-dec-2018.pdf
% https://www.mcafee.com/enterprise/en-us/assets/reports/rp-quarterly-threats-aug-2019.pdf

Digital forensics includes many different aspects, however, the most intrigued part of digital forensics is memory forensics, which "provides unprecedented visibility into the runtime state of the system, such as which processes were running, open network connections, and recently executed commands" as stated in the book The Art of Memory Forensics \cite{ligh2014art}. Not only can we get a frame of a computer state, but we can extract files and processes which was in the memory. If we can identify a malware, we can extract them and learn the malware behaviour through traces in the system. Then we reverse engineer the malware and create its signature and add to our database.

A malware is good only before it is discovered, therefore hackers creates malware hidden from the Task Manager. From the early days of 1990s, hackers have been improving ways to hide malware. In 2017, a small report \cite{evolutionHidding} have revised hiding techniques in malware over the years in Windows OS. [NEED THE PAPER CONCLUSION]. Even in 2019, we can still observe incidences where malware hide itself so effective. For example, an Android malware that hid themselves on user mobile and was only found September 2019 after 5 months on the Google Play Store \cite{hiddenMalwareAndroid}, or the Titanium backdoor\footnote{Programs that receives remote connections} on Windows 10 disclosed November 2019 \cite{titanium}, or the macOS malware, unioncrypto, that download and run a hidden process in memory to mine cryptocurrency was found out December 2019 \cite{unioncrypto}. Researchers collected and analyzed these malware, but a normal user would not be able to do that. For a normal user, if the anti-virus software failed to flag the file as mallicious, his computer will be infected. To know whether a system is having a hidden malware running, we must extract the memory and send to an investigator to find out. Such process is complex, long and costly, a user must know how to extract the memory and hire an investigator to analyze. Another way is to do memory forensics live \footnote{when the system is still running} and automatically extract and send the hidden process binary to the researchers. Shuaibur Rahman and Khan \cite{reviewLive} has a review of live forensics analysis techniques in 2015, one standout paper is \cite{comparativeLive} described the technique to find finished and cache\footnote{background running processes} processes. However, the paper is only restricted to the Linux OS. Others approaches mentioned uses memory acquisition technique to get the memory dump and perform analysis later. There are projects that implements memory forensics techniques but is limited to dump file analysis. We have researched for live memory analysis and have come up with a method for finding hidden processes.

Our target has some requirements, it must be easy to use for normal user, and target Windows 10. The market share of Windows over others OS is higher \cite{osMarketShare}, within the Windows OS, version 10 is now dominating \cite{windowsShare}, also high level of malware targeting Windows has convinced us to set the target to Windows 10. The outcome of this project will be a small tool searching and collecting hidden processes and sending the process binary to the server.

\section[Objectives]{Objectives}

In the scope of this proposal, we wish to:

\begin{itemize}
  \item Understand the basics concept of OS that supports memory forensics.
  \item Understand to some extend the internal of Windows operating system.
  \item Understand some techniques often use for analyzing Windows memory dump.
  \item Analyze some already exists tools support for memory forensics.
  \item Propose a method to find hidden running processes in running Windows machine.
\end{itemize}

\section[Structure]{Structure}

% TODO
